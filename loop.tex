\chapter{循环结构程序设计}
前面介绍了程序中常用到的顺序结构和选择结构,但是只有这两种结构是不够的,还需要用到循环结构。因为在日常生活中或是在程序所处理的问题中常常遇到需要重复处理的问题。例如:

要处理以上问题,最原始的方法是分别编写若干个相同或类似的语句或程序进行处理。

大多数的应用程序都会i包含循环结构。循环和顺序接哦股、选择结构是结构化程序设计的3中基本结构,它们是各种复杂程序的基本构成单元。因此熟练掌握选择结构和循环结构的概念及使用是进行程序设计的最基本的要求。
\section{用while语句实现循环}
则执行while后面的语句(称为循环体)……

这个循环条件就是上面一般形式的“表达式”,它也成为循环条件表达式。

例 求$1+2+3+\cdots +100$。
\section{用do whiel 语句实现循环}
除了while语句以外,C语言还提供了do while语句来实现循环结构。

例 求$1+2+3+\cdots +100$。
\section{用for语句实现循环}
除了可以用while语句和do $\cdots$ while语句实现循环外,C语言还提供for语句实现循环,而且for语句更为灵活,不仅可以用于循环次数已经确定的情况,还可以用于循环次数不确定而只给出循环条件的情况。
\section{循环的嵌套}
一个循环体内包含另一个完整的循环结构,称为循环的嵌套。内嵌的循环中还可以嵌套循环,这就是多层循环。
\section{几种循环的比较}
\begin{enumerate}
	\item 3种循环都可以用来处理同一问题,一般情况下他们可以互相代替。
	\item 在while循环和do while循环中,只在while后面的括号内制定循环条件,因此为了使循环正常结束,应在……
	\item 用while和do while循环时,循环变量初始化的操作应在while和do while语句之前完成。而for语句可以在表达式1中完成。
	\item while循环、do while循环和for循环,都可以用break语句跳出循环,用continue语句结束本次循环。
\end{enumerate}
\section{改变循环执行的状态}
以上介绍的都是根据事先指定的循环条件正常执行和终止的循环。
\subsection{用break语句提前终止循环}
\subsection{用continue语句提前结束本次循环}
有时并不希望终止整个循环的操作,而只是希望提前结束本次循环,而接着执行下次循环。这时可以用continue语句。
\subsection{break语句和continue语句的区别}
\section{循环程序举例}
例 用$\frac{\pi}{4} =1 - \frac{1}{3} + \frac{1}{5}- \frac{1}{7} + \cdots$公式求 $\pi$的近似值,直到发现某一项的绝对值小于 $10^{-6}$为止。

例 求Fibonacci数列的前40个数。这个数列有如下特点:第1,2两个数为1,1.从第3个数开始,该数是其前面两个数之和。

例 输入一个大于3的整数$n$,判定它是否为素数。

例 译密码。
\section{习题}
\begin{enumerate}
	\item 请画出例给出的3个程序段的流程图。
	\item 输入两个正整数$m$和$n$,求其最大公约数和最小公倍数。
	\item 输入一行字符,分别统计出其中英文字母、空格、数字和其他字符的个数。
	\item 求$\sum\limits_{n=1}^{20}n!$
	\item 求$\sum\limits_{k=1}^100 k + \sum\limits_{k=1}^50 k^2+\sum\limits_{k=1}^10 \frac{1}{k}$
	\item 输出所有的“水仙花数”,所谓“水仙花数”是指一个3位数,其各位数字立方和等于该数本身。
	\item 一个数恰好等于它的因子之和,这个数就称为“完数”。编程找出1000之内的所有完数。
	\item 有一个分数序列
		\begin{equation}
			\frac{2}{1},\frac{3}{2},\frac{5}{3},\frac{8}{5},\frac{13}{8},\frac{21}{13}, \cdots
		\end{equation}
		求出这个数列的前20项之和。
	\item 一个球从100cm高度自由落下,每次落地后反跳回原高度的一半,再落下,再反弹。求它在第10次落地时,共经过多少米,第10次反弹多高。
	\item 用迭代法求$x=\sqrt a$。求平方根的迭代公式为
		\begin{equation}
			x_{n+1} = \frac{1}{2}(x_n+\frac{a}{x_n})
		\end{equation}
		要求前后两次求出的$x$的差的绝对值小于$10^{-5}$
	\item 用牛顿迭代法求下面方程在1.5附近的根:
		\begin{equation}
			2x^3-4x^2+3x-6=0
		\end{equation}
	\item 输出以下图案:
		\begin{lstlisting}
		    *
		   ***
		  *****
		 *******
		  *****
		   ***
		    *
		\end{lstlisting}
\end{enumerate}
