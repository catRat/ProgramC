\chapter{善于利用指针}
指针是C语言中一个重要的概念,也是C语言的一个重要特色。正确而灵活地运用它,可以使程序简洁、紧凑、高效。每一个学习和使用C语言的人,都应该深入地学习和掌握指针。可以说,不掌握指针就没有掌握C的精华。

指针的概念比较复杂,使用也比较灵活,因此初学时常会出错,务请在学习本章内容时十分小心,多思考、多比较、多上机,在实践中掌握它。本书在叙述时也力图用通俗易懂的方法使读者易于理解。
\section{指针是什么}
为了所名清楚是指针,必须先清楚数据在内存中是如何存储的,又是如何读取的。

如果在程序中定义了一个变量,在对程序进行编译时,系统就会给这个变量分配内存单元。编译系统根据程序汇总定义的变量类型,分配一定长度的空间。

由于通过地址能找到所需的变量单元,可以说,地址指向变量单元。因此,将地址形象化地称为“指针”。

这种直接按变量名进行访问,称为“直接访问”方式。

还可有采用另一种称为“间接访问”的方式,即变量i的地址存放在另一个变量中,然后通过该变量来找到变量i的地址,从而访问i变量。

在C语言程序中,可以定义整型变量、浮点型变量、字符变量等,也可以定义这样一种特殊的变量,用它存放地址。

一个变量的地址称为该变量的“指针”。
\section{指针变量}
从上节已知:存放地址的变量是指针变量,它用来指向另一个对象。那么,怎样定义和使用指针变量呢?
\section{通过指针引用数组}
先分析一个例子。

例 通过指针变量访问整型变量。
\section{通过指针引用字符串}
在例中已看到怎样定义指针变量,定义指针变量的一般i形式为:
\begin{lstlisting}
类型名 * 指针变量名
\end{lstlisting}

在定义指针变量时必须指定基类型。

一个变量的指针的含义包括两个方面,一是以存储单元编号表示的地址,一是它指向的存储单元的数据类型。
\section{指向函数的指针}
在引用指针变量时,可能有3种情况。
\section{返回指针值的函数}
函数的参数不仅可以是整型、浮点型、字符型等数据,还可以是指针类型。它的作用是将一个变脸的地址传送到另一个函数中。

例 对输入的两个整数按大小顺序输出。用函数处理,且用指针类ixng数据作为函数参数。
\section{通过指针引用数组}
\subsection{数组元素的指针}
一个变量有地址,一个数组包含若干元素,每个数组元素都在内存中占用存储单元,它们都有相应的地址。指针变量既然可以指向变量,当然也可以指向数组元素。所谓ie数组元素的指针就是数组元素的地址。
\subsection{在因哟个数组元素时指针的运算}
\subsection{通过指针引用数组元素}
\subsection{用素组名作函数参数}
\subsection{通过指针引用多维数组}
指针变量可以指向一维数组中的元素,也可以指向多维数组中的元素。但在概念上和使用方法上,多维数组的指针比一维数组的指针要复杂一些。
\begin{enumerate}
	\item 多维数组元素的地址

	\item 指向多维数组元素的指针变量
	\item 用指向数组的指针作函数参数
\end{enumerate}
\section{通过指针引用字符串}
在前几章中已大量地使用了字符串。这些字符串都是以直接方式(字面形式)给出,在一对双撇号中那个包含若干个合法的字符。在本节中将介绍使用字符串的更加灵活方便的方法——通过指针引用字符串。
\subsection{字符串的引用方法}
在C程序中,字符串是存放在字符数组中的。想引用一个字符串,可以用以下两种方法。
\subsection{字符串指针作函数参数}
如果想把一个字符串从一个函数“传递”到另一个函数,可以用地址传递的方法,即用字符数组名作为参数,也可以用字符指针变量做参数。在被调用的函数中可以改变字符串的内容,在主调函数中可以引用改变后的字符串。

用函数实现字符串的复制。
\subsection{使用字符指针变量和字符数组的比较}
用字符数组和字符指针变量都能实现字符串的存储和运算,但它们二者直接按是有区别的,不应混为一谈,主要有以下几点。
\begin{enumerate}
	\item 字符数组由若干个元素组成,每个元素中放一个字符,而字符指针变量中存放的是地址。
	\item 赋值方式。可以对字符指针变量赋值,但不能对数组名赋值。
	\item 初始化的含义。
	\item 存储单元的内容。
	\item 指针变量的值是可以改变的,而数组名代表一个固定的值。
\end{enumerate}
\section{指向函数的指针}
如果在程序中定义了一个函数,在编译时,编译系统为函数代码分配一段存储空间,这段存储空间的起初地址成为这个函数的指针。
\subsection{用函数指针变量调用函数}
如果想调用一个函数,除了可以通过函数名调用以外,还可以通过指向函数的指针变量来调用该函数。

例 用函数求整数a和b中的大者。
\subsection{怎样定义和使用指向函数的指针变量}
\subsection{用只想函数的指针作为函数参数}
\section{返回指针值的函数}
\section{指针数组和多重指针}
一个数组,若其元素均为指针类型数据,称为指针数组,也就是说,指针数组中的没一个元素都存放一个地址,相当于一个指针变量。
\subsection{指针数组作main函数的形参}
指针数组的一个重要应用是作为main函数的形参。
\section{动态内存分配与指向它的指针变量}
\subsection{什么是内存的动态分配}
第7章介绍过全局变量和局部变量,全局变量是分配在内存中的静态存储区,非静态的局部变量是分配在内存中的动态存储区,这个存储区是一个称为栈的区域。
\subsection{怎样建立内存的动态分配}
\subsection{void指针类型}
\section{有关指针的小结}
\section{习题}
