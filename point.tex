\chapter{善于利用指针}
指针是C语言中一个重要的概念,也是C语言的一个重要特色。正确而灵活地运用它,可以使程序简洁、紧凑、高效。每一个学习和使用C语言的人,都应该深入地学习和掌握指针。可以说,不掌握指针就没有掌握C的精华。

指针的概念比较复杂,使用也比较灵活,因此初学时常会出错,务请在学习本章内容时十分小心,多思考、多比较、多上机,在实践中掌握它。本书在叙述时也力图用通俗易懂的方法使读者易于理解。
\section{指针是什么}
为了弄清楚什么是指针,必须先清楚数据在内存中是如何存储的,又是如何读取的。

如果在程序中定义了一个变量,在对程序进行编译时,系统就会给这个变量分配内存单元。编译系统根据程序汇总定义的变量类型,分配一定长度的空间。

由于通过地址能找到所需的变量单元,可以说,地址指向变量单元。因此,将地址形象化地称为“指针”。

这种直接按变量名进行访问,称为“直接访问”方式。

还可有采用另一种称为“间接访问”的方式,即变量i的地址存放在另一个变量中,然后通过该变量来找到变量i的地址,从而访问i变量。

在C语言程序中,可以定义整型变量、浮点型变量、字符变量等,也可以定义这样一种特殊的变量,用它存放地址。

一个变量的地址称为该变量的“指针”。
\section{指针变量}
从上节已知:存放地址的变量是指针变量,它用来指向另一个对象。那么,怎样定义和使用指针变量呢?
\subsection{使用指针变量的例子}
先分析一个例子。

例 通过指针变量访问整型变量。
\subsection{怎样定义指针变量}
在例中已看到怎样定义指针变量,定义指针变量的一般i形式为:
\begin{lstlisting}
类型名 * 指针变量名
\end{lstlisting}

在定义指针变量时必须指定基类型。

一个变量的指针的含义包括两个方面,一是以存储单元编号表示的地址,一是它指向的存储单元的数据类型。
\subsection{怎样引用指针变量}
\subsection{指针变量作为函数参数}
函数的参数不仅可以是整型、浮点型、字符型大亨数据,还可以是指针类型。它的作用是将一个变量的地址传送到另一个函数中。

例 对输入的两个整数按大小顺序输出。用函数处理而且用指针类型的数据作函数参数。

例 对输入的两个整数按大小顺序输出。

例 输入3个整数a,b,c要求按由大到小的顺序将它们输出。用函数实现。
\section{通过指针引用数组}
\subsection{数组元素的指针}
一个变量有地址,一个数组包含若干元素,每个数组元素都在内存中占用存储单元,它们都有相应的地址。指针变量既然可以指向变量,当然也可以指向数组元素。所谓数组元素的指针就是数组元素的地址。
\subsection{在引用数组元素时指针的运算}
\subsection{通过指针引用数组元素}
根据以上叙述,引用一个数组元素,可以用下面两种方法:
\begin{enumerate}
	\item 下标法
	\item 指针法
\end{enumerate}
例 有一个整数数组a,有10个元素,要求输出数组中的全部元素。
\subsection{用过指针引用多维数组}
指针变量可以指向一维数组中的元素,也可以指向多维数组中的元素。
\begin{enumerate}
	\item 多维数组元素的地址
	\item 只想多维数组元素的指针变量
	\item 指向数组的指针作函数参数
\end{enumerate}
\section{通过指针引用字符串}
在前面几章中已大量使用那个了字符串。在本节将介绍使用字符串的更加灵活方便的方法——通过指针引用字符串。
\subsection{字符串的引用方式}
在C程序中,字符串是存放在字符数组中的。想引用一个字符串,可以用以下两种方法。
\begin{enumerate}
	\item 用字符数组存放一个字符串,通过数组名和下标引用字符串中的一个字符,也可以通过数组名和格式生命“\verb|%s|”输出该字符串。
	\item 用字符指针变量指向一个字符串常量,通过字符指针变量引用字符串常量。
\end{enumerate}
\subsection{字符指针作函数参数}
如果想吧一个字符串从一个函数“传递”到另一函数,可以用地址传递的方法,即用字符数组名作参数,也可以用字符指针变量作参数。

例 用函数调用实现字符串的复制。
\subsection{使用字符指针变量和字符数组的比较}
用字符数组和字符指针变量都能实现字符串的存储和运算,但他们二者之间是有区别的,不应混为一谈,主要有一下几点。
\begin{enumerate}
	\item 字符数组由若干个元素组成,每个元素中放一个字符,而字符指针变量中存放的是地址。
	\item 赋值方式。
	\item 初始化的含义。
	\item 存储但那元的内容。
	\item 子真变量的值是可以改变的,而数组名代表一个固定的值,不能改变。
\end{enumerate}


\section{指向函数的指针}
\subsection{什么是函数指针}
如果在程序中定义了一个函数,在编译时,编译系统为函数代码分配一段存储空间,这段存储空间的起始地址称为这个函数的指针。

可以定义一个指向函数的指针变量,用来存放某一函数的起始地址,这就意味着此指针变量孩子先该函数。例如:
\begin{lstlisting}
int (* p)(int, int);
\end{lstlisting}
定义p是一个指向函数的指针变量,它可以只想函数的类型为整型且有两个整型参数的函数。
\subsection{用函数指针调用函数}
如果想调用一个函数,除了可以通过函数名调用意外,还可以通过指向函数的指针变量来调用该函数。

例用函数求整数a和b的大者。
\subsection{怎样定义和使用指向函数的指针变量}
\subsection{用指向函数的指针做函数参数}
指向函数的指针变量的一个重要用途是把函数的地址作为参数传递到其他函数。

指向函数的指针可以作为函数参数,把函数的入口地址传递给形参,这样就能够在被调用的函数中使用实参函数。
在引用指针变量时,可能有3种情况。
\section{返回指针值的函数}
一个函数可以返回一个整型值、字符值、实型值等,也可以返回指针型的数据,即地址。其概念与以前类似,只是返回的值的类型是指针类型而已i。

定义返回指针值的函数的一般形式为:
\begin{lstlisting}
类型名 * 函数名(参数列表);
\end{lstlisting}
例 有a个学生,每个学生有b门课程的成绩。要求在用户输入学生序号以后,能输出该学生的全部成绩。用指针函数来实现。
\section{指针数组和多重指针}
\subsection{什么是指针数组}
\subsection{指向指针数据的指针}
\subsection{指针数组作main函数的形参}

例 对输入的两个整数按大小顺序输出。用函数处理,且用指针类ixng数据作为函数参数。
\section{通过指针引用数组}
\subsection{数组元素的指针}
一个变量有地址,一个数组包含若干元素,每个数组元素都在内存中占用存储单元,它们都有相应的地址。指针变量既然可以指向变量,当然也可以指向数组元素。所谓ie数组元素的指针就是数组元素的地址。
\subsection{在因哟个数组元素时指针的运算}
\subsection{通过指针引用数组元素}
\subsection{用素组名作函数参数}
\subsection{通过指针引用多维数组}
指针变量可以指向一维数组中的元素,也可以指向多维数组中的元素。但在概念上和使用方法上,多维数组的指针比一维数组的指针要复杂一些。
\begin{enumerate}
	\item 多维数组元素的地址

	\item 指向多维数组元素的指针变量
	\item 用指向数组的指针作函数参数
\end{enumerate}
\section{通过指针引用字符串}
在前几章中已大量地使用了字符串。这些字符串都是以直接方式(字面形式)给出,在一对双撇号中那个包含若干个合法的字符。在本节中将介绍使用字符串的更加灵活方便的方法——通过指针引用字符串。
\subsection{字符串的引用方法}
在C程序中,字符串是存放在字符数组中的。想引用一个字符串,可以用以下两种方法。
\subsection{字符串指针作函数参数}

\section{动态内存分配与只想它的指针变量}
\subsection{什么是内存的动态分配}
第7章介绍过全局变量和局部变量,全局变量是分配在内存中的静态存储区,非静态的局部变量是分配在内存中的动态存储区,这个存储区是一个称为栈的区域。
\subsection{怎样建立内存的动态分配}
对内存的动态分配是通过系统的库函数来实现的,主要有malloc,calloc,free,realloc这4个函数。
\begin{enumerate}
	\item 使用malloc函数
	\item 使用calloc函数
	\item 使用free函数
	\item 使用realloc函数
\end{enumerate}
\subsection{void指针类型}
C99允许使用基类型为void的指针类型。可以定义一个类型为void的指针变量,它不知想任何类型的数据。

例 建立动态数组,输入5个学生的成绩,另外ia用一个函数检查其中有无低于60分的,输出不合格的成绩。
\section{有关指针的小结}
由于指针一章介绍的内容较多,指针的概念和应用比较复杂,初学者不易掌握,为了帮组读者建立清晰的概念,本书对有关指针的知识和应用作简单那的归纳小结。
\begin{enumerate}
	\item 首先要准确德弄清楚指针的含义。指针就是地址。
	\item 什么叫“指向”?
	\item 要深入掌握在对数组的操作中正确德使用指针
	\item 指针运算
	\item 指针变量可以有空值
\end{enumerate}
\section{习题}
\begin{enumerate}
	\item 输入3个整数,按由小到大的顺序输出。
	\item 输入3个字符串,按由小到大的顺序输出。
	\item 输入3个整数,按由小到大的顺序输出。
	\item 输入10个整数,将其中最小的数与地一个数对换,把最大的数与最后一个数对换。写3个函数:
		\begin{enumerate}
			\item 输入10个数
			\item 进行处理
			\item 输出10个数
		\end{enumerate}
	\item 有n个整数,使前面各数顺序向后移$m$个位置,最后$m$个数变成最前面$m$个数。写一个函数实现以上功能,在主函数中输入$n$个整数和输出调整后的$n$个数。
	\item 有$n$个人围成一圈,顺序排号。从第1个人开始顺序报号1,2,3.凡报道3者退出圈子。找出最后留在圈子中的人原来的序号。要求用链表实现。
	\item 写一个函数,求一个字符串的长度。在main函数中输入字符串,并输出其长度。
	\item 有一个字符串,包括$n$个字符。写一函数,将此字符串从第$m$个字符开始的全部字符复制成为另一个字符串。
	\item 输入一行文字,找出其中大写字母、小写字母、空格、数字以及其他字符各有多少。
	\item 写一函数,将一个$3 \times 3$的整数矩阵转置。
	\item 将一个$5 \times 5$的矩阵中最大的元素放在中心,4个角分别放4个最小的元素,写一函数实现之。用main函数调用。
	\item 在主函数中输入10个等长的字符串。用另一函数对它们排序。然后在主函数输出这10个已排好序的字符串。
	\item 用指针数组处理上一题目,字符串不等长。
	\item 写一个用矩形法求定积分的通用函数,分别求。
	\item 将$n$个数按输入时顺序的逆序排列,用函数实现。
	\item 有一个班4个学生,5们课程。
	\item 输入一个字符串,内有数字和非数字字符。将其中连续的数字作为一个整数,依次放到一个数组a中。
	\item 写一函数,实现两个字符串的比较。
	\item 编一程序,输入月份号,输出该月的英文月名。
	\item 
		\begin{enumerate}
			\item 编写一个函数new,对n个字符开辟连续的存储空间,此函数返回一个指针,指向字符串开始的空间。new(n)表示分配n个字节的内存空间。
			\item 写一函数free,将前面用new函数占用的空间爱你释放。free(p)表示将p指向的单元以后的内存段释放。
		\end{enumerate}
	\item 用只想指针的指针的方法对5个字符串排序并输出。
	\item 用指向指针的指针的方法对$n$个整数排序并输出。要求将排序单独写成一个函数。$n$个整数在主函数输入,最后在主函数中输出。
\end{enumerate}
