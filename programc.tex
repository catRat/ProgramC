\chapter{FOREWORD}
20世纪90年代以来,C语言迅速在全世界普及推广。无论在中国还是在世界各国,“C语言程序设计”始终是高等学校的一门基本的计算机课程。C语言程序设计在计算机教育和计算机应用中发挥着重要的作用。

作者于1991年编著了……

在此书再版之际,作者想对学习程序设计问题提出以下几点看法。
\section{为什么要学习程序设计?}
大学生,特别是理工科的学生,不能满足于只会用办公软件,应当有更高的要求。

只有懂得程序设计,才能进一步懂得计算机,真正了解计算机是怎样工作的。通过学习程序设计,学会进一步了解计算机的工作原理,更好地理解和应用计算机;掌握计算机处理问题的方法;培养分析问题和解决问题的能力;具有编制程序的初步能力。

因此,无论是算计专业学生还是非计算机专业学生,都应当学习程序设计知识,并且把它作为进一步学习与应用计算机的基础。
\section{为什么选择C语言。}

进行程序设计,必须用一种计算机语言作为工具。可供选择的语言很多。C语言功能丰富、使用灵活方便、应用面广、目标程序效率高、可移植性好。

有人认为C++语言出现后,C语言过时了……它比C语言复杂得多,事实上,将来不是每个人都需要用C++编制大型程序。C语言是更为基本的。美国一位资深软件专家写了……他说“大学生毕业前要学好C语言,C语言是当前程序员共同的语言。它使程序员互相沟通……不管比懂得多少延续、闭包、异常处理,只要你不能解释为什么 \verb| while(* s++ = * t++)|的作用是复制字符串,那你就是在盲目无知的情况下编程,就像一个医生不懂最基本的解剖学就开处方。”

C语言更适合解决某些小型程序的编程。

现在大多数高校把C语言作为第一门计算机语言进行程序设计教学,这是合适的,有了C的基础,在需要时进一步学习C++,也是很容易过度的。
\section{怎样组织程序设计的教学?怎样处理算法和语言的关系。}
进行程序设计,要解决两个问题:
	\begin{enumerate}
		\item 要学习和掌握解决问题的思路和方法,即算法;
		\item 学习怎样实现算法,即用计算机语言编写程序,达到用计算机解题的目的。
	\end{enumerate}

因此,课程的内容应当主要包括两个方面:算法和语言……作者的做法是:以程序设计为中心,把二者紧密结合起来,既不能孤立地抽象研究算法,也不能孤立地学习语法。

算法是重要的,但本科从不是专门研究算法与逻辑的理论课程,不可能系统全面地介绍算法;也不是脱离语言环境研究算法,而是在学习编程过程中,介绍有关典型算法,引导学生思考怎样构造一个算法。编写程序的过程就是i设计算法的过程。

语言工具也是重要的……决不能把程序设计课程变成枯燥地介绍语法的课程,学校语法要服务于编程。

我们坚决摒弃孤立地介绍语法的做法,而是以程序设计为中心,把算法与语言紧密结合起来。不是根据语言规则的分类和顺序作为教学和教材的章节和孙需,而是从应用的角度切入,以编程为目的,从初学者的认识规律出发,由浅入深,构造教材和教学的体系……随着变成难道的逐步提供,算法和语法的学习同步趋于深入。学生在富有创意的编程中,学会了算法,掌握了语法,把枯燥无味的语法规则变成生动活泼的编程应用。事实证明这种做法是成功的。

近年来许多学校的经验表明,按照这种思路进行教学,取到教师容易教,学生容易学的效果。
\section{怎样学习C程序设计}
	\begin{enumerate}
		\item 要着眼于培养能力。C语言程序设计并不是一门纯理论的课程,而是一门应用的课程。应当注意培养分析问题的能力、构造算法的能力、编程的能力和调试程序的能力。
		\item 要把重点放在解题的思路上,通过大量的例题学习怎样设计一个算法,构造一个程序。初学时不要在语法细节上死抠。语法细节是需要通过较长的实践才能掌握的。
		\item 掌握基本要求,注意打好基础。如果学时有限,有些内容可以选学,把精力放在最基本、最常用的内容上,学好基本功。
		\item 要十分重视实践环节。光靠听课和看书是学不会程序设计的,学习本课程既要掌握概念,又必须动手变成,还要亲自上机调试运行。读者一定要重视实践环节,包括编程和上机……考核方法应当是编写程序和调试长线,而不应该只采用选择题。
		\item 要举一反三。学校程序设计,主要是掌握程序设计的思路和方法。学会i使用一种计算机语言变成,在需要时改用另一种语言应该不会太困难。
		\item 要提倡和培养创新精神。教师和学生都不应该局限于教材中的内容,应该启发学生的学习兴趣和创新意识。能够在教材程序的基础上,思考更多的问题,编写难度更大的程序。在本书每章的习题中,包括了一些难度较大的题目,建议学生尽量选做,学会自己发展只是,提高能力。
		\item 如果对学生有较高的程序设计要求,应当在学习本课程后,安排一次集中的课程设计环节,要求学生独立完成一个有一定规模的程序。
	\end{enumerate}
\section{从实际出发,区别对待}
对计算机专业学生……
\section{为什么要修订《C程序设计》}
任何工作都要与时俱进……
\section{为了满足不同的需要,出版不同层次的C程序设计教材}
\chapter{常见错误分析}
