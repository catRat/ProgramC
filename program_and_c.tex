\chapter{程序设计和C语言}

\section{什么是计算机程序}

第一个要明确的概念是计算机程序(程序)。程序是用来告诉计算机对数据如何进行处理的指令集合。

\section{什么是计算机语言}

这种计算机能直接识别和而接受的二进制代码称为机器指令(machine instruction)。机器指令的集合就是该计算机的机器语言(machine language)。

符号语言

人们创造了符号语言,它用一些英文字母和数字表示一个指令,符号语言又称为符号汇编语言或汇编语言(assembler language)。

高级语言的发展历程
高级语言经历了不同的发展阶段:
\begin{enumerate}
	\item 非结构化的语言。初期的语言属于非结构化的语言,编程风格比较随意,只要符合语法规则即可,没有严格的规范要求,程序中的流程可以随意跳转。人们为为最修程序执行的效率而采用许多“小技巧”,使程序变得难于阅读和维护。
	\item 结构化的语言。为了解决以上问题,提出了“结构化程序设计方法”,规定程序必须由具有良好特性的基本结构(顺序结构、分支结构、顺环结构)构成,程序中的流程不允许随意跳转,程序总是由上而下顺序执行各个基本结构。这种程序结构清晰,易于编写、阅读和维护。QBASIC,FORTRAN 77和C语言等属于结构化的语言,这些语言的特点是支持结构化程序设计方法。

以上两种语言都是基于过程的语言,在编写程序时需要具体指定每一个过程的细节。在编写规模较小的程序时,还能得心应手,但在处理规模较大的程序时,就显得捉襟见肘、力不从心了。在实践的发展中,人们又提出额面向对象的程序设计方法。程序面对的不是过程的细节,而是一个对象,对象是有数据以及对数据进行的操作组成的。
	\item 面向对象的语言。近十多年来,在处理规模较大的问题时,开始使用面向对象的语言。\verb|C++,C#| ,Visual Basic和Java等语言是支持面向对象程序设计方法的语言。有关面向对象的程序设计方法和面向对象的语言在本书不做详细介绍,有兴趣的可参考有关专门书籍。
\end{enumerate}

\section{C语言的发展及其特点}

本书是介绍怎样利用C语言作为工具进行程序设计的。为什么要选择C语言呢? 这里有必要对C语言的发展和特点有一定的了解。

C语言是国际上广泛流行的计算机高级语言。

C语言的祖先是BCPL语言。1967年英国剑桥大学的Martin Richards推出没有类型的BCPL(Basic Combined Programming Languade)语言。1970年美国AT\&T贝尔实验室的Ken Thompson以BCPL语言为基础,设计除了很简单且很接近硬件的B语言(取BCPL的第一个字母)。但B语言过于简单,功能有限。1972-1973年间,美国贝尔实验室的D.M.Ritchie在B语言的基础上设计出了C语言。C语言既保持了BCPL和B语言的优点(精练,接近硬件),又克服了它们的缺点(过于简单、无数据类型等),C语言的新特点主要表现在具有多种数据类型(如字符、数值、数组、结构体和指针等)。开发C语言的目的在于尽可能降低用它写的软件对硬件平台的依赖程度,使之具有可移植性。

最初的C语言只是为描述和实现UNIX操作系统提供一种工作语言而设计。1973年,Ken Thompson和D.M.Ritchie合作把UNIX的90\%以上用C语言改写,即UNIX第5版。随着UNIX的日益广泛使用,C语言也迅速得到推广。1978年以后,C语言先后移植到大、中、小和微型计算机上。C语言便很快风靡全世界,成为世界上应用最广泛的程序设计高级语言。

以UNIX第7版中的C语言编译程序为基础,1978年,Brian W.Kernighan和Dennis M.Ritchie合著了影响深远的名著The C Programming Language,本书中介绍的C语言成为广泛使用的C语言版本的基础,它是实际上第一个C语言标准。

C语言是一种用途广泛、功能强大、使用灵活的过程性(procedural)编程语言,即可用于编写应用软件,又能用于编写系统软件。因此C语言问世以来得到迅速推广。自20世纪90年大初,C语言在我国开始推广以来,学习和使用C语言的人越来越多,成了学习和使用人数最多的一种计算机语言,绝大多数理工科大学都开设了C语言程序设计课程。掌握C语言成为计算机开发人员的一项基本功。

C语言有以下一些主要特点。
\begin{enumerate}
	\item 语言简洁、紧凑,使用方便、灵活。C语言一共只有37个关键字,9种控制语句,程序书写形式自由。C语言程序比其他许多高级语言简练,源程序短,因此输入程序时工作量少。
	\item 运算符丰富。C语言的运算符包含的范围很广泛,共有34种运算符。C语言把括号、赋值和强制类型转换等都作为运算符处理,从而使C语言的运算类型极其丰富,表达式类型多样化。灵活使用各种运算符可以实现在其他高级语言种难以实现的运算。
	\item 数据类型丰富。C语言提供的数据类型包括:整型、浮点型、字符型、数组类型、指针类型、结构体类型和共用体类型等,C99又扩充了复数浮点类型、超长整型和布尔类型等。尤其时指针类型数据,使用十分灵活和多样化,能用来事项各种复杂的数据结构(如链表、树、栈等)的运算。
	\item 具有结构化的控制语句。用函数作为程序的模块单位,便于事项程序的模块化。C语言时完全模块化和结构化的语言。
	\item 语法限制不太严格,程序设计自由度大。
	\item C语言允许直接访问物理地址,能进行位(bit)操作,能实现汇编语言的大部分功能,可以直接对硬件进行操作。
	\item 用C语言编写的程序可移植性好。
	\item 生成目标代码质量高,程序执行效率高。
\end{enumerate}

C原来是专门为编写系统软件而设计的,许多大的软件都用C语言编写,这是因为C语言的可移植性好和硬件控制能力高,表达和运算能力强。许多以前只能用汇编语言处理的问题,后来可以改用C语言来处理了。目前C的主要用途之一是编写“嵌入式系统程序”。由于具有上述优点,使C语言应用十分广泛,许多应用软件也用C语言编写。

对C语言以上的特点,待学完C语言以后再回顾以下,就会有比较深的体会。

\section{最简单的C语言程序}

为了使用C语言编程序,必须了解C语言,并且能熟练地使用C语言。本书将由浅到深地介绍怎样阅读C语言程序和使用C语言编写程序。

\subsection{最简单的C语言程序举例}

\subsection{C语言程序的结构}
\section{运行C程序的步骤与方法}

 计算机不能直接识别和执行用高级语言写的指令。必须用编译程序把C源程序翻译成二进制形式的目标程序,然后再将该目标程序与系统的函数库以及其他目标程序连接起来,形成可执行的目标程序。

 在编写好一个C源程序后,怎样上级进行编译和运行呢?一般要经过以下几个步骤:
\begin{enumerate}
	\item 	上机输入和编辑源程序。
	\item 对源程序进行编译,先用C编译系统提供的“预处理器”对程序中的预处理指令进行编译预处理。由预处理得到的信息与程序其他部分一起,组成一个完整的、可以用来进行正是编译的源程序,然后由编译系统对该源程序进行编译。

 编译的作用首先是对源程序进行检查,判定它有无语法方面的错误,如有,则发出“出错信息”,告诉编程人员认真检查改正。修改程序后重新进行编译,如有错,在发出“出错信息”。如此反复进行,知道没有语法错误为止。这是,编译程序自动把源程序转换为二进制形式的目标程序。如果不特别指定,此目标程序一般也存放在用户当前目录下,此时源文件没有消失。

 在用编译系统对源程序进行编译时,自动包括了预编译和正式编译两个阶段,一气呵成。用户不必分别发出二次指令。

	\item 进行链接处理。经过编译所得到的二进制文件还不能共计算机直接执行。前面已说明:一个程序可能包含若干个源程序文件,而编译是以源程序文件为对象的,一次编译只能得到与一个源程序文件相对应的目标文件,它只是整个程序的一部分。必须把所有的编译后得到的目标模块链接装配起来,在于函数库相连接成为一个整体,生成一个可供计算机执行的目标程序,成为可执行程序(executive program)。

 即使一个程序只包含一个源程序文件,编译后得到的目标程序也不能直接运行,也要经过连接阶段,因为要与寒暑假进行链接,才能生成可执行程序。
	\item 运行可执行程序,得到运行结果。
\end{enumerate}

 一个程序从编写到运行成功,并不是一次成功的,往往要经过多次反复。编写好的程序并不一定能保证正确无误,除了用人工方式检查外,还必须借助编译系统来检查有无语法错误。

 为了编译、链接和运行C程序,必须要有相应的编译系统。

 写出源程序后可以用任何一种编译系统对程序进行编译和连接工作,只要用户感到方便、有效即可。

 对编译系统的态度是,不应当只会一种编译系统,无论用哪一种编译系统,都应当能举一反三,在需要时会用其他编译系统进行工作。

\section{程序设计的任务}

如果只是编写和运行一个很简单的程序,上面介绍的步骤就够了。但是实际上要处理的问题比上面见到的例子复杂得多,需要考虑和处理的问题也复杂得多。程序设计就是指从确定任务到得到结果、写出文档的全过程。
从确定问题到最后完成任务,一般经历以下几个工作阶段:
\begin{enumerate}
	\item 问题分析。对于接手的任务要进行认真的分析,研究所给定的条件,分析最后应达到的目标,找出解决问题的规律,选择解体的方法。在此过程种可以忽略一些次要的因素而使问题抽象化。这就是建立模型。
	\item 设计算法。即设计出解体的方法和具体步骤。
	\item 编写程序。根据得到的算法,用一种高级语言编写出源程序。
	\item 对源程序进行编辑、编译和链接,得到可执行程序。
	\item 行程序,分析结果。运行可执行程序,得到运行结果。能得到运行结果并不意味着程序正确,要对结果进行分析,看它是否合理。
	\item 编写程序文档。许多程序使提供给别人使用的,如同正式的产品应当提供产品说明书一样,正式提供给用户使用的程序,必须向用户提供程序说明书。内容包括:程序名称、程序功能、运行环境、程序的装入和启动给、需要输入的数据,以及使用的注意事项等。
\end{enumerate}
\section{习题}
\begin{enumerate}
	\item 什么是程序?什么是程序设计?
	\item 为什么需要计算机语言?高级语言的特点?
	\item 正确理解以下名词及其含义:
		\begin{enumerate}
			\item 源程序、目标程序、可执行程序
			\item 程序编辑、程序编译、程序链接
			\item 程序、程序模块、程序文件
			\item 函数、主函数、被调用函数、函数库
			\item 程序调试、程序测试
		\end{enumerate}
	\item 自学本书中附录A,熟悉上机运行C程序的方法,上级运行本章3个例题。
	\item 请参照本章例题,编写一个C程序,输出以下信息:
	\item 编写一个C程序,输入a,b,c三个值,输出其中最大者。
	\item 上级运行以下程序,注意注释的方法。分析运行结果,掌握注释的用法。
		\begin{lstlisting}
int main()
{
    printf("How do you do!\n"); //这是行注释,注释范围从//至换行符
}
printf("How do you do!\n"); /*这是块注释*/
printf("How do you do!\n"); /*这是
			      块注释*/
	        \end{lstlisting}
\end{enumerate}
