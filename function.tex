\chapter{用函数实现模块化程序设计}
通过前几章的学习,已经能够编写一些简单的C程序了,但是如果程序的功能比较多,规模比较大,把所有的程序代码都写在一个主函数中,就会使主函数变得庞杂。此外,有时程序中要多次实现某一功能,就需要多次重复编写此功能的程序代码。

因此,人们自然会想到采用“组装”的方法来简化程序设计的过程。如同组装计算机一样。这就是模块化程序设计的思路。

在设计一个较大的程序时,往往把它分为若干个程序模块,每一个模块包括一个或多个函数,每个函数实现一个特定的功能。一个C程序可由一个主函数和若干个其他函数构成。由主函数调用其他函数,其他函数也可以相互调用。

除了可以使用库函数外,有的部门还编写一批本领域或本单位常用到的专门函数,供本领域或本单位的人员使用。在程序设计中要善于利用函数,以减少重复编写程序段的工作量,也更便于是新建模块化的程序设计。

例 想输出一下的结果,用函数调用实现。
\begin{lstlisting}
*******************
How dy you do!
*******************
\end{lstlisting}
\section{怎样定义函数}
\subsection{为什么要定义函数}
C语言要求,在程序中用到的所有函数,必须“先定义,后使用”。

定义函数应包括一下几个内容:
\begin{enumerate}
	\item 制定函数的名字,以便以后按此名调用。
	\item 制定函数的类型,即函数返回值的类型。
	\item 制定函数的参数的名字和类型,以便在调用函数时向它们传递数据。
	\item 制定函数应当完成什么操作,也就是函数是做什么的,即函数的功能。这是最重要的,是在函数体中解决的。
\end{enumerate}
对于C编译系统提供的库函数,是由编译系统事先定义好的,库文件中包括了对各函数的定义。程序设计者不必自己定义。在有关的头文件中包括了对函数的声明。库函数只提供了最基本、最通用的一些函数,而不可能包括人们在实际应用中所用到的所有库函数。程序设计者需要在程序中自己定义想用的而库函数并没有提供的函数。
\subsection{定义函数的方法}
\begin{enumerate}
	\item 定义无参函数
	\item 定义有参函数
	\item 定义空函数
\end{enumerate}
\section{调用函数}
定义函数的目的是为了调用此函数,以得到预期的结果。因此,应当熟练掌握调用函数的方法和有关概念。
\subsection{函数调用的形式}
调用一个函数的方法很简单,如前面已见过的。

按函数掉调用在程序中出现的形式和位置来分,可以有以下3中函数调用方式。
\begin{enumerate}
	\item 函数调用语句
	\item 函数表达式
	\item 函数参数
\end{enumerate}
\subsection{函数调用时数据传递}
\begin{enumerate}
	\item 形式参数和实际参数

		在调用有参函数时,主调函数和被调函数之间有数据传递关系。
	\item 实参和形参间的数据传递
\end{enumerate}

例 输入两个整数,要求输出其中较大者。要求用函数来找到大数。
\subsection{函数调用的过程}
\begin{enumerate}
	\item 在定义函数中指定的形参,在未出现函数调用时,它们并不占内存中的存储单元。在发生函数调用时,函数max的形参被临时分配内存单元。
	\item 将实参对应的值传递给形参。
	\item 在执行max函数期间,由于形参已经有值,就可以利用形参进行有关的运算。
	\item 通过return语句将函数值带回到主调函数。
	\item 调用结束,形参单元被释放。
\end{enumerate}
\subsection{函数的返回值}
通常,希望通过函数调用使主调函数能得到一个确定的值,这就是函数值。

下面对函数值作一些说明。
\begin{enumerate}
	\item 函数的返回值是通过函数中的return语句获得的。
	\item 函数值的类型
	\item 在定义函数时指定的函数类型一般应该和return语句中的表达式类型一致。
	\item 对于不带返回值的函数,应当用定义函数为“void类型”。
\end{enumerate}
\section{对被调用函数的声明和函数原型}
在一个函数中调用另一个函数需要具有如下条件:
\begin{enumerate}
	\item 首先被调用的函数必须是已经定义的函数。但仅有这一条件还不够。
	\item 如果使用库函数,应该在本文件开头用指定将调用有关库函数所需要用到的信息包含到本文件中。
	\item 如果是哟你该用户自己定义的函数,而该函数的位置在调用它的函数的后面,应该在主调函数中对被调用的函数作声明。
\end{enumerate}

例 输入两个实数,用一个函数求出它们之和。
\section{函数的嵌套调用}
C语言的函数定义是互相平行、独立的,也就是说,在定义函数时,一个函数内不同再定义另一个函数,也就是不同嵌套定义,但可以嵌套调用函数。

例 输入4个整数,找出其中最大的数。用函数的嵌套调用来处理。
\section{函数的递归调用}
在调用一个函数的过程中出现直接或间接地调用该函数本身,称为函数的递归调用。C语言的特点之一在于允许函数的递归调用。

例 用递归的方法求$n!$。
\section{数组作为函数参数}
调用有参函数时,需要提供实参。实参可以是常量、变量或表达式。数组元素的作用与变量相当。因此,数组元素也可以用作函数实参,其用法与变量相同。此外,数组名也可以作实参和形参,传递的是数组第一个元素的地址。
\subsection{数组元素作为函数实参}
数组元素可以用作函数实参,不能用作形参。因为形参是在函数被调用时临时分配存储单元的,不可能我ie一个数组元素单独分配存储单元。在用数组元素作函数实参时,把实参的值传递给形参,是“值传递”方式。数据的传递方向是从实参传递到形参,单向传递。

例 输入10个数,要求输出其中最大的元素和该数是第几个数。
\subsection{数组名作函数参数}
除了可以用数组元素作为ie函数参数外,还可有用数组名作函数参数。应当注意的是:哟个数组元素作实参时,想形参变量传递的是数组元素的值,而用数组名作函数实参时,向形参传递的是数组首元素的地址。

例 有一个一维数组score,内放10个学生成绩,求平均成绩。

例 有两个班级,分别有35名和30名学生,调用一个average函数,分别求这两个班的学生的平均成绩。

例 用选择法对数组中10个整数按由小到大排序。
\subsection{多维数组名作函数参数}
多维数组元素可以作函数惨胡,这点与前述的情况类似。

例 有一个$3 \times 4$的矩阵,求所有元素中的最大值。
\section{局部变量和全局变量}
在学习本章之前见到的程序大多数是一个程序只包含一个main函数,变量是在函数的开头定义的。这些变脸挂在本函数范围内有效,即在本函数开头定义的变量,在本函数中可以被引用。

这就是变量的作用域问题。没一个变量都有一个作用域问题,即它们在什么范围内有效。
\subsection{局部变量}
定义变量可能有3中情况:
\begin{enumerate}
	\item 在函数的开头定义;
	\item 在函数内的复合语句内定义;
	\item 在函数的外部定义。
\end{enumerate}
\subsection{全局变量}
前已介绍,程序的编译单位i是源程序文件,一个源文件可以包含一个或若干个函数。自爱函数内定义的变量是局部变量,而在函数之外定义的变量称为外部变量,外部变量是全局变量。

例 有一个一维数组,内放10个学生成绩,写一个函数,当主函数调用此函数后,能求出平均分、最高分和最低分。
\section{变量的存储方式和生存期}
\subsection{动态存储方式与静态存储方式}
从上一节已知,从变量的作用域的角度来观察,变量可分为全局变量和局部变量。

还可以从另一个角度,即从变量值存在的时间来观察。
\subsection{局部变量的存储类别}
\begin{enumerate}
	\item 自动变量
	\item 静态局部变量
	\item 寄存器变量
\end{enumerate}
\subsection{全局变量的存储类别}
\section{关于变量的声明和定义}
\section{内部函数和外部函数}
变量有作用域,有局部变量和外部变量之分,那么函数有没有类似的问题呢?有的,有的函数可以被本文件中的其他函数调用,也可以被其他文件中的函数调用,而有的函数只能被本文件中的其他函数调用,不能被其他文件中的函数调用。

函数在本质上是全局的,因为定义一个函数目的就是要被另外的函数调用……但是,也可以制定某些函数不能被其他文件调用。根据函数能否被其他源文件调用,将函数区分为内部函数和外部函数。
\subsection{内部函数}
\subsection{外部函数}
\section{习题}
