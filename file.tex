\chapter{对文件的输入输出}
\section{C文件的有关基本知识}
凡是用过计算机的人都不会对“文件”感到模式,大多数人都接触过或使用过文件。在程序中使用文件之前应该了解有关文件的基本知识。
\subsection{什么是文件}
文件有不同的类型,在程序设计中,主要用到的两种文件:
\begin{enumerate}
	\item 程序文件。包括源程序文件、目标文件……这种文件的内容是程序代码。
	\item 数据文件。文件的内容不是程序,而是供程序运行时读写的数据,如在程序运行过程输出到磁盘的数据,或在程序运行过程中供读入的数据。
\end{enumerate}
本章主要讨论的是数据文件。

在以前各章中所处理的数据的输入和输出,都是一终端为对象的,即从终端的键盘输入数据,运行结果输出到终端显示器上。实际上,常常需要将一些数据输出到磁盘上保存起来,以后需要时再从磁盘中输入到计算机内存。这就要用到磁盘文件。

为了简化用户对输入输出设备的操作,使用户不必去区分各种输入输出设备直接按的区别,操作系统吧各种设备都统一作为文件来处理。从操作系统哦你搞得角度看,每一个与主机相连的输入输出设备都看作一个文件。

文件(file)是程序设计中一个重要的概念。所谓“文件”一般值存储在外部介质上数据的集合。一批数据是一文件的形式存放在外部介质上的。操作西欧他那个是一文件为单位对数据进行管理的,也就说,如果想找村发给你在外部介质上的数据,必须先按文件名找到所制定的文件,然后再从该文件中读取数据。要向外部介质上存储数据也必须先建立一个文件,才能想它输出数据。

输入输出是数据传送的过程,数据如流水一样从一处流向另一处,因此常将输入输出形象地称为流(stream),即数据流。流表示了信息从源到目的端的流动。在输入操作时,数据从文件流向 计算机内存,在输出操作时,数据从计算机流向文件。文件是由运行环境进行统一管理的,无论用Word打开或保存文件,还是C程序中的输入输出都是通过操作系统进行的。“流”是一个传输通道,数据可以从运行唤醒流入程序中,或从程序流至运行环境。

从C程序的观点来看,无论程序一次读写一个字符,或一行文字,或一个制定的数据区,作为输入输出的各种文件或设备都是统一以逻辑数据流的方式出现的。C语言把文件看作是一个字符的序列,即由一个一个字符的数据顺序组成。一个输入输出流就是一个字符流或字节流。

C的数据文件由一连串的字符组成,而不考虑行的界限,两行数据间不会自动加分隔符,对文件的存取是以字符(字节)为但问的。输入输出数据流的开始和结束仅受程序空孩子而不受物理负号控制,这就增加了处理的灵活性。这种文件称为流式文件。
\subsection{文件名}
一个文件要有一个唯一的文件标识,以便用户识别和引用。文件标识包括3部分:(1)文件路径;(2)文件名主干;(3)文件后缀。

为了方便起见,文件标识常被称为文件名,但应了解此时所称的文件名,实际上包括以上3部分内容,而不仅是文件名主干。文件名主干的命名规则遵循标识符的命名规则。
\subsection{文件的分类}
根据数据的组织形式,数据文件可分为ASCII文件和二进制文件。数据在内存中是以二进制形式存储的,如果不加转换地输出到外村,就是二进制文件,可以认为它就是存储在内存的数据的映像,所以也称之为映像文件(image file)。如果要求在外存上以ASCII代码形式存储,则需要在存储前进行转换。ASCII文件又称为文本文件(text file),每一个字节放一个字符的ASCII代码。

一个数据在磁盘上怎样存储呢?字符一律以ASCII形式存储,数值型数据既可以用ASCII形式存储,也可以用二进制形式存储。

用ASCII码形式输出时字节与字符一一对应,一个字节代表一个字符,因而便于对字符进行逐个处理,也便于输出字符。但一般占存储空间较多,而且要花费转换时间。用二进制形式输出数值,可以节省外存空间和转换时间,把内存中的存储单元中的内容远封不动地输出到磁盘上,此时每一个字节并不一定代表一个字符。

\subsection{文件缓冲区}
ANSI C标准采用“缓冲文件系统”处理数据文件,所谓缓冲文件系统是指系统自动地在内存为程序中每一个正在使用的文件开辟一个文件缓冲区。从内存向磁盘输出数据必须先送到内存中的缓冲区,装满缓冲区后在一起送到磁盘去。如ugo从磁盘想计算机读入数据,则一次从磁盘文件将一批数据输入到内存缓冲区,然后再从缓冲区诸葛地将数据送到程序数据区。缓冲区的大小由各个具体的C编译系统确定。
\subsection{文件类型指针}
缓冲文件系统中,关键的概念是“文件类型指针”,简称“文件指针”。每个被使用的文件都在内存中开辟一个相应的文件信息区,用来存放文件的有关信息(如文件的名字、文件状态以及文件当前位置等)。这些信息是保存在一个结构体便狼中。该结构提类新故事由系统声明的,取名为FILE。

不同C编译系统的FILE类型包含的内容不完全相同,但大同小异。
\section{打开与关闭文件}
对文件读写之前应该“打开”该文件,在使用结束之后应“关闭”该文件……所谓“打开”是指为文件建立相应的信息区和文件缓冲区。

在编写程序时,在打开文件的同时,一般都制定一个指针变脸个直线该文件,也就是建立起指针变脸关于文件之间的联系,这样,就可以通过该指针变量对文件进行读写了。所谓“关闭”是指撤销文件信息区和文件缓冲区,是文件指针变量不再指向该文件,显然就无法进行对文件的读写了。
\subsection{用fopen函数打开数据文件}
\subsection{用fclose函数关闭数据文件}
\section{顺序读写数据文件}
文件打开之后,就可以对它进行读写了。在顺序写时,先写入的数据存放在文件中前面的位置,后写入的数据存放在文件中后面的位置。在顺序读时,先读文件中前面的数据,厚度文件中后面的数据。也就是说,对顺序读写来说,对文件读写数据的顺序和数据在文件中的物理顺序是一致的。顺序读写需要用库函数实现。
\subsection{怎样想文件读写字符}
对文件读入和输出一个字符的函数。
\begin{table}
	\begin{tabular}{l}
		\hline
		fgetc \\
		fputc \\
		\hline
	\end{tabular}
\end{table}
\subsection{用格式化的方式读写文件}
前面进行的是字符的输入输出,而实际上数据的类型是丰富的。大家已很熟悉用printf函数和scanf函数向终端进行格式化的输入输出。其实也可以对文件爱你进行格式化输入输出,这时就要用fprintf函数和fscanf函数,从函数名可以看到,它们只是在printf和scanf的前面加一个字母“f”。它们的作用与printf函数和scanf函数相仿,都是格式化读写函数。只有一点不同:。它们的一般调用方式为
\begin{lstlisting}
fprintf(filename, str, list);
fscanf(fiee, str, list);
\end{lstlisting}
\subsection{用二进制方式向文件爱你读写一组数据}
在程序中不仅需要一次输入输出一个数据,而且常常需要一次输入输出一组数据,C语言允许用freed函数从文件中读一个数据块,用fwrite函数向文件写一个数据块。自爱读写时时以二进制形式进行的。在向磁盘写数据时,直接向内存中一组数据原封不动地复制到磁盘文件上,在读入时也是将磁盘文件中若干字节的内容一批读入内存。

它们的一般调用形式为
\begin{lstlisting}
fread(buffer, size, count, fp);
fwrite(buffer, size, count, fp);
\end{lstlisting}
\section{随机读写数据文件}
对文件进行顺序读写比较容易理解,也容易操作,但是有时效率不高。
\subsection{文件位置标记及其定位}
\subsection{文件位置标记的定位}
\section{文件读写的出错检测}

