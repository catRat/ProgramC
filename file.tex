\chapter{对文件的输入输出}
\section{C文件的有关基本知识}
凡是用过计算机的人都不会对“文件”感到模式,大多数人都接触过或使用过文件。在程序中使用文件之前应该了解有关文件的基本知识。
\subsection{什么是文件}
文件有不同的类型,在程序设计中,主要用到的两种文件:
\begin{enumerate}
	\item 程序文件。包括源程序文件、目标文件……这种文件的内容是程序代码。
	\item 数据文件。文件的内容不是程序,而是供程序运行时读写的数据,如在程序运行过程输出到磁盘的数据,或在程序运行过程中供读入的数据。
\end{enumerate}
本章主要讨论的是数据文件。

在以前各章中所处理的数据的输入和输出,都是一终端为对象的,即从终端的键盘输入数据,运行结果输出到终端显示器上。实际上,常常需要将一些数据输出到磁盘上保存起来,以后需要时再从磁盘中输入到计算机内存。这就要用到磁盘文件。

为了简化用户对输入输出设备的操作,使用户不必去区分各种输入输出设备直接按的区别,操作系统把各种设备都统一作为文件来处理。从操作系统的角度看,每一个与主机相连的输入输出设备都看作一个文件。

文件(file)是程序设计中一个重要的概念。所谓“文件”一般值存储在外部介质上数据的集合。一批数据是一文件的形式存放在外部介质上的。操作系统是以文件为单位对数据进行管理的,也就说,如果想找存放在外部介质上的数据,必须先按文件名找到所指定的文件,然后再从该文件中读取数据。要向外部介质上存储数据也必须先建立一个文件,才能想它输出数据。

输入输出是数据传送的过程,数据如流水一样从一处流向另一处,因此常将输入输出形象地称为流(stream),即数据流。流表示了信息从源到目的端的流动。在输入操作时,数据从文件流向计算机内存,在输出操作时,数据从计算机流向文件。文件是由运行环境进行统一管理的,无论用Word打开或保存文件,还是C程序中的输入输出都是通过操作系统进行的。“流”是一个传输通道,数据可以从运行环境流入程序中,或从程序流至运行环境。

从C程序的观点来看,无论程序一次读写一个字符,或一行文字,或一个制定的数据区,作为输入输出的各种文件或设备都是统一以逻辑数据流的方式出现的。C语言把文件看作是一个字符的序列,即由一个一个字符的数据顺序组成。一个输入输出流就是一个字符流或字节流。

C的数据文件由一连串的字符组成,而不考虑行的界限,两行数据间不会自动加分隔符,对文件的存取是以字符(字节)为单位的。输入输出数据流的开始和结束仅受程序空孩子而不受物理负号控制,这就增加了处理的灵活性。这种文件称为流式文件。
\subsection{文件名}
一个文件要有一个唯一的文件标识,以便用户识别和引用。文件标识包括3部分:(1)文件路径;(2)文件名主干;(3)文件后缀。

文件路径表示文件在外部存储设备中的位置。

为了方便起见,文件标识常被称为文件名,但应了解此时所称的文件名,实际上包括以上3部分内容,而不仅是文件名主干。文件名主干的命名规则遵循标识符的命名规则。后缀用来表示文件的性质,一般不超过3个字母,如doc。
\subsection{文件的分类}
根据数据的组织形式,数据文件可分为ASCII文件和二进制文件。数据在内存中是以二进制形式存储的,如果不加转换地输出到外存,就是二进制文件,可以认为它就是存储在内存的数据的映像,所以也称之为映像文件(image file)。如果要求在外存上以ASCII代码形式存储,则需要在存储前进行转换。ASCII文件又称为文本文件(text file),每一个字节放一个字符的ASCII代码。

一个数据在磁盘上怎样存储呢?字符一律以ASCII形式存储,数值型数据既可以用ASCII形式存储,也可以用二进制形式存储。

用ASCII码形式输出时字节与字符一一对应,一个字节代表一个字符,因而便于对字符进行逐个处理,也便于输出字符。但一般占存储空间较多,而且要花费转换时间。用二进制形式输出数值,可以节省外存空间和转换时间,把内存中的存储单元中的内容远封不动地输出到磁盘上,此时每一个字节并不一定代表一个字符。

\subsection{文件缓冲区}
ANSI C标准采用“缓冲文件系统”处理数据文件,所谓缓冲文件系统是指系统自动地在内存为程序中每一个正在使用的文件开辟一个文件缓冲区。从内存向磁盘输出数据必须先送到内存中的缓冲区,装满缓冲区后在一起送到磁盘去。如果从磁盘向计算机读入数据,则一次从磁盘文件将一批数据输入到内存缓冲区,然后再从缓冲区诸葛地将数据送到程序数据区。缓冲区的大小由各个具体的C编译系统确定。
\subsection{文件类型指针}
缓冲文件系统中,关键的概念是“文件类型指针”,简称“文件指针”。每个被使用的文件都在内存中开辟一个相应的文件信息区,用来存放文件的有关信息(如文件的名字、文件状态以及文件当前位置等)。这些信息是保存在一个结构体变量中。该结构体类型是由系统声明的,取名为FILE。

不同C编译系统的FILE类型包含的内容不完全相同,但大同小异。
\section{打开与关闭文件}
对文件读写之前应该“打开”该文件,在使用结束之后应“关闭”该文件。“打开”和“关闭”是形象的说法,好像打开闷才能进入房子,门关闭就无法进入一样。实际上,所谓“打开”是指为文件建立相应的信息区和文件缓冲区。

在编写程序时,在打开文件的同时,一般都制定一个指针变量指向该文件,也就是建立起指针变量关于文件之间的联系,这样,就可以通过该指针变量对文件进行读写了。所谓“关闭”是指撤销文件信息区和文件缓冲区,是文件指针变量不再指向该文件,显然就无法进行对文件的读写了。
\subsection{用fopen函数打开数据文件}
ANSI C规定了用标准输入输出函数fopen来实现打开文件。

fopen函数的调用方式为
\begin{lstlisting}
fopen("a1", "r");
\end{lstlisting}
\subsection{用fclose函数关闭数据文件}
在使用完一个文件后应该关闭它,以防止它再被误用。“关闭”就是撤销文件信息区和文件缓冲区,使文件指针变量不再指向该文件,也就是文件指针变量与文件“托钩”,此后不能再通过该指针与其相联系的文件进行读写操作,除非再次打开,使该指针变量重新指向该文件。

关闭文件用fclose函数。fclose函数调用的一般形式为
\begin{lstlisting}
fclose(fp);
\end{lstlisting}

如果不关闭文件将会丢失数据。因为,在向文件写数据时,是先将数据输出到缓冲区,待缓冲区充满后才正式输出给文件。如果当数据未充满缓冲区而程序结束运行,就有可能使缓冲区的数据丢失。要用fclose函数关闭文件,先把缓冲区中的数据输出到磁盘文件,然后才能撤销文件信息去。有的编译系统在程序结束前会自动先将缓冲区中的数据写道文件,从而避免了这个问题,但还是应当养成在程序终止之前关闭所有文件的习惯。

fclose函数也带回一个值,当成功地执行了关闭操作,则返回值为0;否则返回EOF。
\section{顺序读写数据文件}
文件打开之后,就可以对它进行读写了。在顺序写时,先写入的数据存放在文件中前面的位置,后写入的数据存放在文件中后面的位置。在顺序读时,先读文件中前面的数据,后读文件中后面的数据。也就是说,对顺序读写来说,对文件读写数据的顺序和数据在文件中的物理顺序是一致的。顺序读写需要用库函数实现。
\subsection{怎样向文件读写字符}
对文件读入和输出一个字符的函数见表。
\begin{table}
	\begin{tabular}{l}
		\hline
		fgetc \\
		fputc \\
		\hline
	\end{tabular}
\end{table}
例 从键盘输入一些字符,逐个把他们送到磁盘上去,知道用户输入一个“\#”为止。

例 将一个磁盘文件中的信息复制到另一个磁盘文件中。今要求将上例建立的file.dat文件的内容复制到另一个磁盘文件file2.dat中。
\subsection{怎样向文件读写一个字符串}
前面已掌握了向磁盘文件读写一个字符的方法,有的读者很自然地提出一个问题,如果字符个数多,一个一个读和写太麻烦,能否一次读写一个字符串。

C语言允许通过函数fgets和fputs一次读写一个字符串,例如:
\begin{lstlisting}
fgets(str, n, fp);
\end{lstlisting}
作用是从fp所指向的文件中读入一个长度为$n-1$的字符串,并在最后加一个\verb|'\0'|字符,然后把这n个字符放在字符数组str中。

例 从键盘读入若干个字符串,对它们按字母大小的顺序排序,然后把排序好的字符串送到磁盘文件中保持。
\subsection{用格式化的方式读写文件}
前面进行的是字符的输入输出,而实际上数据的类型是丰富的。大家已很熟悉用printf函数和scanf函数向终端进行格式化的输入输出。其实也可以对文件进行格式化输入输出,这时就要用fprintf函数和fscanf函数,从函数名可以看到,它们只是在printf和scanf的前面加一个字母“f”。它们的作用与printf函数和scanf函数相仿,都是格式化读写函数。只有一点不同:fprintf和fscanf函数的读写对象不是终端而是文件。它们的一般调用方式为
\begin{lstlisting}
fprintf(filename, str, list);
fscanf(fiee, str, list);
\end{lstlisting}
\subsection{用二进制方式向文件爱你读写一组数据}
在程序中不仅需要一次输入输出一个数据,而且常常需要一次输入输出一组数据,C语言允许用freed函数从文件中读一个数据块,用fwrite函数向文件写一个数据块。自爱读写时时以二进制形式进行的。在向磁盘写数据时,直接向内存中一组数据原封不动地复制到磁盘文件上,在读入时也是将磁盘文件中若干字节的内容一批读入内存。

它们的一般调用形式为
\begin{lstlisting}
fread(buffer, size, count, fp);
fwrite(buffer, size, count, fp);
\end{lstlisting}

例 从键盘输入10个学生的有关数据,然后把他们转村到磁盘文件上去。
\section{随机读写数据文件}
对文件进行顺序读写比较容易理解,也容易操作,但是有时效率不高,例如文件中有1000个数据,若只查第1000个数据,必须先逐个读入前面999个数据,才能读入第1000个数据。如果文件中存放一个城市几百万人的资料,若按此方法查某一人的情况,等待的时间可能是不能忍受的。

随机访问不是按数据在文件中的物理位置次序进行读写,而是可以对任何位置上的数据进行访问,显然这种方法比顺序访问效率高得多。
\subsection{文件位置标记及其定位}
123
\begin{enumerate}
	\item 文件位置标记

		前已介绍,为了对读写进行控制,系统为每个文件设置了一个文件读写位置标记,用来指示“接下来读写的下一个字符的位置”。

		一般情况下,在对字符文件进行顺序读写时,文件位置标记指向文件开头,这时如果对文件进行读操作,就读第1个字符,然后文件位置标记向后移一个位置,在下一次执行读的操作时,就将位置标记指向第2个字符读入。以此类推,直到遇到文件结尾,结束。

		如果是顺序写文件……
	\item 文件位置标记的定位

		可以强制使文件位置标记只想人们制定的位置。可以用以下函数实现。
		\begin{enumerate}
			\item 用rewind函数使文件位置标记指向文件开头。
			\item fseek
		\end{enumerate}
	\item ……
\end{enumerate}
\subsection{随机读写}
\section{文件读写的出错检测}
C提够一些函数用来检查输入输出函数调用时可能出现的错误。
\begin{enumerate}
	\item ferror函数
	\item clearerr函数
\end{enumerate}
\section{习题}
\begin{enumerate}
	\item 什么是文件型指针?通过文件指针访问文件有什么好处?
	\item 对文件的打开与关闭的含义是什么?为什么要打开和关闭文件?
	\item 从键盘输入一个字符,将其中的小写字母全部转换成大写字母,然后输出到一个磁盘文件“test”中保存,输入的字符串以“!”结束。
	\item 有两个磁盘文件“A”和“B”,各存放一行字母,今要求把这两个文件中的信息合并,输出到一个新文件“C”中去。
	\item ……
\end{enumerate}

