\chapter{利用数组处理批量数据}
第5章之前的程序中使用的变量都属于基本类型,例如整型、字符型、浮点型数据,这些 都是简单的数据类型。对于简单的问题,使用这些简单的数据类型就可以了。但是,对于有些需要处理的数据,只用以上简单的数据类型是不够的,难一反映出数据的特点,也难以有效地进行处理。

这个右下标的数字称为下标(subscript)。一批具有同名的同属性的数据就组成一个数组(array),s就是数组名。

由此可知:
\begin{enumerate}
	\item 数组是一组有序数据的集合。
	\item 用一个数组名和下标来唯一地确定数组中的元素。
	\item 数组中的没一个元素都属于同一数据类型。
\end{enumerate}

将数组和循环结合起来,可以有效地处理大批量的数据,大大提高了工作效率。

本章介绍在C语言中怎样使用数组来处理同类型的批量数据。
\section{怎样定义和引用一维数组}
一维数组是数组中最简单的,它的元素只需要用数组名加一个下标……
\subsection{怎样定义一维数组}
要使用数组,必须在程序汇总先定义数组,即通知计算机:由那些数据组成数组,数组中有多少元素,属于哪个数据类型。

定义一维数组的一般形式为:
\begin{lstlisting}
类型符 数组名[常量表达式]
\end{lstlisting}
\subsection{怎样引用一维数组元素}
在定义数组并对其中各元素赋值后,就可以引用数组中的元素。
\subsection{一维数组的初始化}
为了使程序简洁,常在定义数组的同时,给各数组元素赋值,这称为数组的初始化。可以用“初始化列表”方法失信啊数组的初始化。
\subsection{一维数组程序举例}
例 用数组处理求Fibonacci数列问题。
\section{怎样定义和引用二维数组}
前面已提到,有的问题需要用二维数组来处理。

二维数组常称为矩阵(matrix)。把二维数组写成行(row)和列(column)的排列形式,可以有助于形象化地理解二维数组的逻辑结构。
\subsection{怎样定义二维数组}
怎样定义二维数组呢?基本概念与方法和一维数组相似。如:
\begin{lstlisting}
float pay[3][6];
\end{lstlisting}
以上定义了一个float型二维数组。
\subsection{怎样引用二维数组的元素}
\subsection{二维数组的初始化}
\subsection{二维数组程序举例}
例 将一个二维数组的行和列的元素互换,存到另一个二维数组中。
\section{字符数组}
前已介绍:字符型数据是以字符的ASCII代码存储在存储单元中的,一般占一个字节。由于ASCII代码也属于整数形式,因此在C99标准中,把字符类型归纳为整型类型中的一种。

由于字符数据的应用较广泛,尤其是作为字符串形式使用,有其自己的特点,因此,在本书中专门加以讨论,希望读者熟练地掌握。

C语言中没有字符串类型,字符串是存放在字符型数组中的。
\subsection{怎样定义字符数组}
\subsection{字符数组的初始化}
\subsection{怎样引用字符数组中的元素}
例 输出一个已知的字符串。

例 输出一个菱形。
\subsection{字符串和字符串结束的标志}
\subsection{字符数组的输入输出}
\subsection{使用字符串处理函数}
在C函数库中提供了一些用来专门处理字符串的函数,使用方便。几乎所有版本的C语言编译器都提供这些函数。下面介绍集中常用的函数。
\begin{enumerate}
	\item puts函数——输出字符串的函数
	\item gets函数——输入字符串的函数
	\item strcat函数——字符串连接函数
	\item strcpy和strncpy函数——字符串复制函数
	\item strcmp函数——字符串比较函数
	\item strlen函数——测字符串长度的函数
	\item strlwr函数——转换为小写的函数
	\item strupr函数——转换为大写的函数
\end{enumerate}
\subsection{字符数组应用举例}
例 输入一行字符,统计其中有多少单词,单词之间用空格隔开。

例 有3个字符串,要求找出其中最大者。
\section{习题}
\begin{enumerate}
	\item 用筛选法求100之内的素数。
	\item 用选择法对10个整数排序。
	\item 求一个$3 \times 3$的整型矩阵对角线元素之和。
	\item 有一个已排好序的数组,要求输入一个数后,按原来排序的规律将它插入数组中。
	\item 将一个数组中的值按逆序重新存放。
	\item 输出一下的杨辉三角形
	\item 输出“魔方阵”。所谓魔方阵是这样的方阵,它的每一行,每一列和对角线之和均相等。
	\item 找出一个二维数组中的鞍点,即该位置上的元素在该行上最大、在该列上最小。也可能没有鞍点。
	\item 有15个数按有大到小顺序存放在一个数组中,输入一个数,要求用折半查找方法找出该数是数组汇总第几个元素的值。如果该数不在数组中,则输出“无此数”。
	\item 有一篇文章,共有3行文字,每行80个字符。要求分别统计出其中英文大写字母、小写字母、数字、空格以及其他字符的个数。
	\item 输出以下图案:
		\begin{lstlisting}
*****
 *****
  *****
   *****
    *****
     *****
		\end{lstlisting}
	\item 有一行电文,已按下面规律译成密码:
		\begin{lstlisting}
A -> Z a -> z
B -> Y b -> y
C -> X c -> x
		\end{lstlisting}
		要求编程将密码译回原文,并输出密码和原文。
	\item 编一程序,将两个字符串连接起来,不要用strcat函数。
	\item 编一个程序,将两个字符串s1和s2比较。
	\item 编写一个程序,将字符数组s2中的全部字符复制到字符数组s2中。不用strcpy函数。复制时,\verb|'\0'|也要复制过去。
\end{enumerate}
