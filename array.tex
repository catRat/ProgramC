\chapter{利用数组处理批量数据}
第5张之前的程序中使用的变量都属于基本类型,例如整型、字符型、浮点型数据,这些 都是简单的数据类型。对于简单的问题,使用这些简单的数据类型就可以了。但是,对于有些需要处理的数据,只用以上简单的数据类型是不够的,难一反映出数据的特点,也难以有效地进行处理。

这个右下标的数字称为下标(subscript)。一批具有同名的同属性的数据就组成一个数组(array),s就是数组名。

由此可知:
\begin{enumerate}
	\item 数组是一组有序数据的集合。
	\item 用一个数组名和下标来唯一地确定数组中的元素。
	\item 数组中的没一个元素都属于同一数据类型。
\end{enumerate}

将数组和循环结合起来,可以有效地处理大批量的数据,大大提高了工作效率。

本章介绍在C语言中怎样使用数组来处理同类型的批量数据。
\section{怎样定义和引用一维数组}
一维数组是数组中最简单的,它的元素只需要用数组名加一个下标……
\subsection{怎样定义一维数组}
要使用数组,必须在程序汇总先定义数组,即通知计算机:由那些数据组成数组,数组中有多少元素,属于哪个数据类型。

定义一维数组的一般形式为:
\begin{lstlisting}
类型符 数组名[常量表达式]
\end{lstlisting}
\subsection{怎样引用一维数组元素}
在定义数组并对其中各元素赋值后,就可以引用数组中的元素。
\subsection{一维数组的初始化}
为了使程序简洁,常在定义数组的同时,给各数组元素赋值,这称为数组的初始化。可以用“初始化列表”方法失信啊数组的初始化。
\subsection{一维数组程序举例}
例 用数组处理求Fibonacci数列问题。
\section{怎样定义和引用二维数组}
前面已提到,有的问题需要用二维数组来处理。

二维数组常称为矩阵(matrix)。把二维数组写成行(row)和列(column)的排列形式,可以有助于形象化地理解二维数组的逻辑结构。
\subsection{怎样定义二维数组}
\subsection{二维数组程序举例}
\section{字符数组}
前已介绍:字符型数据是以字符的ASCII代码存储在存储单元中的,一般占一个字节。由于ASCII代码也属于整数形式,因此在C99标准中,把字符类型归纳为整型类型中的一种。

由于字符数据的应用较广泛,尤其是作为字符串形式使用,有其自己的特点,因此,在本书中专门加以讨论,希望读者熟练地掌握。

C语言中没有字符串类型,字符串是存放在字符型数组中的。
\subsection{怎样定义字符数组}
\subsection{字符数组的初始化}
\subsection{怎样引用字符数组中的元素}
\subsection{字符串和字符串结束的标志}
\subsection{字符数组的输入输出}
\subsection{使用字符串处理函数}
\subsection{字符数组应用举例}
\section{习题}
