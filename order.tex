\chapter{最简单的C程序设计——顺序程序设计}

有了前两章的基础,现在可以开始由浅入深地学习C语言程序设计了。

为了能编写C语言程序,必须具备以下地知识和能力:
\begin{enumerate}
	\item 要有正确地解题思路,即学会设计算法,否则无从下手。
	\item 掌握C语言的语法,知道怎样使用C语言所提供的功能编写出一个完整的、正确的程序。也就是在设计好算法之后,能用C语言正确表示此算法。
	\item 在写算法和编写程序时,要采用结构化程序设计方法,编写出机构化的程序。
\end{enumerate}
算法的种类很多,可以说是无底洞,不可能等到把所有的算法都学透以后再来学习编程序。C语言的语法规定很多、很烦琐,独立地学习语法不但枯燥乏味,而且即使倒背如流,也不一定能写出一个好的程序。必须找到一种有效的学习方法。

本书的做法是:以程序设计为主线,把算法和语法紧密结合起来,引导读者由易到及难地学会编写C程序。对于简单的程序,算法比较简单,程序中涉及的语法现象也比较简单。对于比较复杂的算法,程序中用到的语法现象也比较复杂(例如要使用数组、指针和结构体等)。本章先从简单的程序开始,介绍简单的算法,同时介绍最基本的语法现象,使读者具有编写简单的程序的能力。在此基础上,逐步介绍复杂一些的程序,介绍比较复杂的算法,同时介绍深入的语法现象,把算法与语法有机地结合起来,不不深入,由简单到复杂,使读者很自然地、循序渐进地学会编写程序
\section{顺序程序设计举例}
例3.1 有人用温度计测量出用华氏法表示的温度,今要求把它转化为以摄氏法表示的温度。
\begin{equation}
	c = \frac{5}{9}(f - 32)
\end{equation}

例3.2 计算存款利息。有1000元,想存一年。有3种方法:(1)火气,年利率为$r1$;(2)一年期定期,年利率为$r2$;(3)存两次半年定期,年利率为$r3$。请分别计算出一年后按3中方法所的到的本息和。
\section{数据的表现形式及其运算}
有了以上写程序的基础,本节对程序中最基本的成分作必要的介绍。
\subsection{常量和变量}
在计算机高级语言中,数据有两种表现形式:常量和变量。
\begin{enumerate}
	\item 常量

		在程序运行过程中,其值不能被改变的两称为常量。

		常用的常量有以下几类:
		\begin{enumerate}
			\item 整型常量。
			\item 实型常量。
			\item 字符常量。
			\item 字符串常量。
			\item 符号常量。
		\end{enumerate}
	\item 变量

		变量代表一个有名字、具有特定属性的一个存储单元。它用来存放数据,也就是存放变量的值。在程序运行期间,变量的值是可以改变的。

		变量必须先定义,后使用。在定义时制定改变量的名字和类型。一个变量应该有一个名字,以便被引用。
	\item 常变量

		C99允许使用常变量,如
		\begin{lstlisting}
			const int a = 3;
		\end{lstlisting}
		表示a被定义为一个整形变量,指定其值为3,而且在变量存在期间其值不能改变。
	\item 标志符

		在计算机高级语言中,用来对变量、符号常量名、函数、数组、类型等命名的有效字符序列统称为标识符(identifier)。
\end{enumerate}
\subsection{数据类型}
C语言要求在定义所有变量的时候要制定变量的类型。

所谓的类型,就是对数据分配存储单元的安排,包括存储单元的长度以及数据的存储形式。不同的类型分配不同的长度和存储形式。

\subsection{整形类型}
\begin{enumerate}
	\item 整型数据的分类

		本节介绍基本的整形类型。
		
		\begin{enumerate}
			\item int
			\item short int
			\item long int
			\item long long int
		\end{enumerate}
	\item 整型变量的符号属性
\end{enumerate}
\subsection{字符型数据}
由于字符是按其代码形式存储的,因此C99把字符型数据作为整数类型的一种。但是,字符型数据在使用上有自己的特点,因此把它单独列为一节来介绍。
\begin{enumerate}
	\item 字符与字符代码
	\item 字符变量
\end{enumerate}
\subsection{浮点型数据}
浮点型数据用来表示具有小数点的实数。
\begin{enumerate}
	\item float
	\item double
	\item long double
\end{enumerate}
\subsection{怎样确定常量的类型}
怎样确定常量的类型呢?从常量的表示形式即可以判断其类型。

整型常量

浮点型常量
\subsection{运算符和表达式}
几乎每一个程序都需要进行计算,对数据进行加工处理,否则程序就没有意义了。要进行运算,就需规定可以是使用的运算符。C语言的运算符范围很宽,把除了控制语句和输入输出以外的几乎所有的基本操作都作为运算符处理,如方括号,点号。
\begin{enumerate}
	\item 基本的算术运算符
	\item 自增、自减运算符
	\item 算术表达式和运算符的优先级与结合性

		用运算符和括号将运算对象(操作数)链接起来、符合C语法规则的式子,成为C算术表达式。
	\item 不同类型数据间的混合运算

		这些转换是编译系统自动完成的,用户不必过问。
	\item 强制类型转换运算符
	\item C运算符

		\begin{itemize}
			\item 算术运算符	\verb|+- * / ++ --|
			\item 关系运算符	\verb|> < == >= <= !=|
			\item 逻辑运算符 	\verb|! && |
			\item 位运算符		\verb|<<>>~ |
			\item 赋值运算符=及其扩展赋值运算符
			\item 条件运算符	\verb|?:|
			\item 都好运算符	\verb|,|
			\item 指针运算符	\verb|* &|
			\item 求字节数运算符	sizeof
			\item 强制类型转换运算符	(类型)
			\item 成员运算符	\verb|. ->|
			\item 下表运算符	\verb|[]|
			\item 其他  ()
		\end{itemize}
\end{enumerate}

例 给定一个大写字母,要求用小写字母输出。

\section{C语句}
\subsection{C语句的作用和分类}
在前面的例子可以看到:一个函数包含声明部分和执行部分,执行部分是由语句组成的,语句的作用是想计算机系统发出操作指令,要求执行相应的操作。一个C语句经过编译后产生若干条机器指令。声明部分不是语句,它不产生机器指令,只是对有关数据的声明。

C程序的结构可以用图表示。即一个C程序可以由若干个源程序文件组成,一个源文件可以由若干个函数和预处理指令以及全局变量声明部分组成。一个函数由数据声明部分和执行语句组成。

C语句分为以下5类。
\begin{enumerate}
	\item 控制语句
	\begin{itemize}
		\item if()...else
		\item 2)for()...
		\item while()...
		\item do...while
		\item continue
		\item break
		\item switch
		\item return 
		\item goto
	\end{itemize}
	\item 函数调用语句。函数调用语句由一个函数调用加上一个分号构成。
	\item 表达式语句。
	\item 空语句。
	\item 复合语句。
\end{enumerate}
\subsection{最基本的语句——赋值语句}
在C程序中,最常用的语句是:赋值语句和输入输出语句。其中最基本的是赋值语句。程序中的计算功能大部分是由赋值语句实现的。

例 给出三角形的长,求三角形的面积。

\section{数据的输入输出}
\subsection{输入输出举例}
前面已经看到了利用printf函数进行数据输出的程序,现在再介绍一个包含输入和输出的程序。

例 求$ax^2 + bx + c = 0$方程的根。$a, b, c$由键盘输入,设$b^2 - 4ac > 0$。
\subsection{有关数据输入输出的概念}
从前面的程序可以看到:几乎没一个C程序都包含输入输出。因为要进行运算,就必须给出数据,而运算的结果当然需要输出,一般人们应用。没有输出的程序是没有意义的,输入输出是程序中最基本的操作之一。

在讨论程序的输入输出时首先要注意一下几点。
\begin{enumerate}
	\item 所谓输入输出是以计算机为主体而言的。
	\item C语言本身不提供输入输出语句,输入和输出操作是由C标准函数库中的函数来实现的。在C标准函数库中提供了一下诶输入输出函数。

		C提供的标准函数以库的形式在C的编译系统中提供,它们不是C语言文本中的组成部分
\end{enumerate}
\subsection{用printf函数输出数据}
\subsection{用scanf函数输入数据}
\subsection{字符数据的输入输出}
除了可以用printf函数和scanf函数输出和输入字符外,C函数库还提供了一些专门用于输入和输出字符的函数。他们是很容易理解和使用的。
\begin{enumerate}
	\item putchar
	\item getchar
\end{enumerate}

\section{习题}
\begin{enumerate}
	\item 假如我国国民生产总值的年增长率为9\%,计算10年后我国国民生产总值与现在相比增长多少百分比。计算公式为
		\begin{equation}
			p = (1 + r)^n
		\end{equation}
		$r$为年增长率,$n$为年数,$p$为与现在相比的倍数。
	\item 存款利息的计算。有100元,想存5年,可按以下5种方法存:
	\item 购房从银行贷了一笔款$d$,准备每月还款额为$p$,月利率为$r$,计算多少月能还清。设$d$为300000元,$P$为6000元,$r$为1\%。对求得的月份取小数点后一位,对第2位按四舍五入处理。
	\item 分析下面的程序:
		\begin{lstlisting}
char c1, c2;
c1 = 97;
c2 = 98;
printf("c1 = %c, c2 = %c\n", c1, c2);
printf("c1 = %d, c2 = %d\n", c1, c2);
		\end{lstlisting}
		\begin{enumerate}
			\item 运行时会输出什么信息?为什么?
			\item 如果将程序第4,5行改为
				\begin{lstlisting}
c1 = 197;
c2 = 198;
				\end{lstlisting}
				运行时会输出什么信息?为什么?
			\item 如果将程序第3行改为
				\begin{lstlisting}
int c1, c2;
				\end{lstlisting}
				运行时会输出什么信息?为什么?
		\end{enumerate}

	\item 用下面的scanf函数输入数据,使\verb|a=3, b=7, x=8.5,y=71.82,c1='A',c2='a'|。问在键盘上如何输入?
		\begin{lstlisting}
int a, b;
float x, y;
char c1, c2;
scanf("a=%db=%d", &a, &b);
scanf("%fb=%e", &a, &y);
scanf("%c%c", &c1, &c2);
		\end{lstlisting}
	\item 请便程序将“China”译成密码,密码规律是:用原来的字母后面第4个字母代替原来的字母。
	\item 设元半径$r=1.5$,圆柱高$h=3$,求圆周长、园面积、圆球面积、圆球体积、圆柱体积。用scanf输入数据,输出计算机结果,输出时要求有文字说明,取小数点后2位。请编写程序。
	\item 编程序,用getchar函数读入两个字符c1和c2,然后分别用putchar函数和printf函数输出这两个字符。思考以下问题:
		\begin{enumerate}
			\item 变量c1和c2应定义为字符型还是整型?或二者皆可?
			\item 要求输出c1和c2指的ASCII码,应如何处理?用putchar函数还是printf函数?
			\item 整型变量与字符变量是否在任何情况下都可以互相代替?是否无条件地等价?
		\end{enumerate}
		
\end{enumerate}
%	\item 写两个函数,分别求两个整数的最大公约数和最小公倍数,用主函数调用这两个函数,并输出结果。两个整数由键盘输入。
%	\item 求方程 $ax^2 + bx + c = 0$ 的根,用3个函数分别求当:$b^2 - 4ac$ 大于0、等于0和小于0时的根并输出结果。
%	\item 写一个判素数的函数,在主函数输入一个整数,输出是否为素数的信息。
%	\item 写一个函数,使给定一个$3 \times 3$的二维整型数组转置,即行列互换。
%	\item 写一个函数,使输入的一个字符串按反序存放,在主函数中输入和输出字符串。
